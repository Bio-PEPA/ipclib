\documentclass[10pt,a4paper]{article}
\usepackage{amssymb,amsbsy,verbatim,fancybox,pepa}
\usepackage{graphicx}
\usepackage{verbatim}

\newcommand{\adcComment}[1]{\textbf{#1}}
\newcommand{\adcCommentl}[1]{\textbf{#1}\\}
\newcommand{\rewordthis}[1]{\textsl{#1}\\}

\newcommand{\ipc}{\textsf{ipc}}
\newcommand{\hydra}{\textrm{Hydra}}
\newcommand{\Condor}{\textrm{Condor}}
\newcommand{\hydrus}{\textsf{hydrus}}
\newcommand{\collate}{\textsf{collate}}
\newcommand{\puffball}{\textsf{puffball}}
\newcommand{\gnuplot}{\textsf{gnuplot}}
\newcommand{\pepa}{\textrm{PEPA}}
\newcommand{\sectref}[1]{Sect.~\ref{#1}}
\newcommand{\figref}[1]{Fig.~\ref{#1}}
%

\title{Fix for Passage-Time Probes}
\author{
Allan Clark
}


\begin{document}
\maketitle


\section{Introduction}
This document is intended to describe the fix for a major bug that was found in
the \ipc\cite{pepahydra} \pepa\cite{pepa} compiler.

\subsection{The Bug}
To compute performance measures in general \ipc\ attaches
a performance measurement probe.
A probe is defined by the user to be a sequence of activities.
Some of the activies may be labelled. By default \ipc\ will measure between
the two labels \texttt{start} and \texttt{stop}.

When doing passage-time analysis the idea is that we wish to compute the
probability of exiting the passage of interest (defined as being between
observations of the labelled actions of the probe) a given time after
entering. To do this a master probe was added in addition to the
control probe. This master probe has two states, \texttt{running} and
\texttt{stopped}.
It is a very simple \pepa\ component which moves from the \texttt{stopped}
to the \texttt{running} state whenever the \texttt{start} label is observed
and moves from the \texttt{running} to the \texttt{stopped} state whenever
the \texttt{stop} label is observed.

The \hydra\ markov chain analyser accepts performance measurement specifications
as conditions on the state of the model.
For passage-time analysis one must provide a starting condition
and an ending condition.

This meant that the given performance measurement to \hydra\ was the probe
being in the \texttt{running} state for the starting condition and the probe
being in the \texttt{stopped} state for the terminating condition.

The problem with this is that hydra accumulates the passage-time probabilities
considering all the states of the model in which the probe is in the
\texttt{running} state.
In other words it sees them all the intermediate states
between the start and end of the passage as also being start states.

\section{The Solution}
The solution is to add a further probe which we will call the passage-probe.
This probe also has two states. We call these two states the \texttt{waiting}
state and the \texttt{starting} state.
The idea is that the probe is in the \texttt{starting} state only for all those
states which are immediately after the start of the passage of interest.
The cooperation set is all the activities in the model including the communication
messages sent by any probes added by the user.

To achieve this the probe must cooperate over the composition of the model
and the normal master probe. When in the \texttt{waiting} state the passage-probe
self-loops on all activities except those which being the passage of interest.
By default this will be the single $start$ communication message performed by the
master probe.

\section{Disadvantages}
I think it breaks wherever we have a diminishing state immediately after
the the starting state of the model since in this case I think the passage-probe
may never get into the starting state.
This could perhaps be solved by self-looping on immediate actions within the
\texttt{starting} state.

Additionally every action in the model becomes a cooperation which may slow
down compile times.


\section{Results}
I used the model shown in Figure
\ref{figure:erlangpepa}.
This is a simple straight-line model with no cooperation or choice
so the results should be predictiable. In fact we use it to compute
an erlang function over the number of actions performed in sequence
(all at the same rate).

\begin{figure}
\verbatiminput{../erlang.pepa}
\caption{
\label{figure:erlangpepa}
Input pepa model used for computing erlang graphs
}
\end{figure}

\newcommand{\graphicfigure}[3]{
\begin{figure}
\includegraphics[scale=0.5]{#1}
\caption{
\label{#2}
#3
}
\end{figure}
}

\graphicfigure{../pdfk.pdf}
              {figure:erlangpdfk}
              {The pdfs with increasing k values.}

\graphicfigure{../cdfk.pdf}
              {figure:erlangcdfk}
              {The cdfs with increasing k values.}

\graphicfigure{../ratepdf.pdf}
              {figure:ratepdfs}
              {The pdf of an erlang of length eight at varying lengths}

\graphicfigure{../ratecdf.pdf}
              {figure:ratecdfs}
              {The cdf of an erlang of length eight at varying lengths}

% \includegraphics[scale=0.5]{../pdfk.png}

% \includegraphics[scale=0.5]{../cdfk.png}

% \includegraphics[scale=0.5]{../ratepdf.png}

% \includegraphics[scale=0.5]{../ratecdf.png}

%%%%%%%%%%%%%%%%%%%%%%%%%%%%%%%%%%%%%%%%%%%%%%%%%%%%%%%%%%%%%%%%%%%%
\bibliographystyle{alpha}
\bibliography{pepa}
\end{document}

