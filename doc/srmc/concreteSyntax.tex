The concrete syntax for srmc model files is given in Figure
\ref{figure:syntax:srmcfiles}.
The parts of the grammar which are written down within an srmc file
are highlighted in \textcolor{blue}{blue}, everything else is a 
bnf style operator.

%%%%%%%%%%%%%%%%%%%%%%%%%%%%%%%%%%%%%%%%%%%%%%%%%%%%%%%%%%
%% General commands for writing a grammar
\newcommand{\esyntaxlabel}[1]{
( #1 )
}

\newcommand{\esyntaxtopline}[3]{
$ #1 $ & $ := $ & $ #2 $ & \esyntaxlabel{ #3 }
}
\newcommand{\esyntaxline}[2]{
& $ \mid $ & $ #1 $ &  \esyntaxlabel{ #2 }
}


%%%%%%%%%%%%%%%%%%%%%%%%%%%%%
% The commands for the grammar
\newcommand{\model}{ model }
\newcommand{\definitionList}{def\_list}
\newcommand{\definition}{def}

\newcommand{\ratedef}{rate\_def}
\newcommand{\processdef}{process\_def}
\newcommand{\namespace}{name\_space}

% Rates
\newcommand{\rateId}{rate\_ident}
\newcommand{\qualifiedRateId}{Q\rateId}
\newcommand{\rateSet}{rate\_set}
\newcommand{\rate}{rate}
\newcommand{\rateExp}{rate\_exp}
\newcommand{\commaRateExps}{rate\_list}
\newcommand{\rateOp}{rOper}
\newcommand{\rateCond}{rCond}


% Processes
\newcommand{\processId}{process\_ident}
\newcommand{\qualifiedProcessId }{ Q\processId }
\newcommand{\processSet}{process\_set}
\newcommand{\process}{process}
\newcommand{\commaProcesses}{process\_list}
\newcommand{\prefix}{prefix}
\newcommand{\condOper}{condOper}

% Conditional behaviours
\newcommand{\beCondition}{condition}

% Name spaces
\newcommand{\namespaceId}{space\_ident}
\newcommand{\qualifiedNameSpaceId}{Q\namespaceId}

% Actions
\newcommand{\actionName}{action\_name}
\newcommand{\actionList}{action\_list}

% Numbers
\newcommand{\forgivFloat}{int\_or\_float}

%Identifiers
\newcommand{\lowerid}{lower\_ident}
\newcommand{\upperid}{upper\_ident}
\newcommand{\qualifier}{qualifier}
\newcommand{\qualifiedLower}{Q\lowerid}
\newcommand{\qualifiedUpper}{Q\upperid}

\newcommand{\lowers}{\concrete{a}-\concrete{z}}
\newcommand{\uppers}{\concrete{A}-\concrete{Z}}
\newcommand{\digits}{\concrete{0}-\concrete{9}}

\newcommand{\systemEquation}{system}

% bnf operators
\newcommand{\optional}[1]{\langle #1 \rangle}
\newcommand{\oneOrMore}[1]{ #1^+ }
\newcommand{\bnfBracket}[1]{ \{ #1 \} }
\newcommand{\oneOrMoreBracketed}[1]{\oneOrMore{\bnfBracket{#1}}}
\newcommand{\zeroOrMore}[1]{ #1^* }
\newcommand{\zeroOrMoreBracketed}[1]{\zeroOrMore{\bnfBracket{#1}}}


%Parts of the actual concrete grammar
\newcommand{\concrete}[1]{\textcolor{blue}{\texttt{#1}}}
\newcommand{\concreteBraces}[1]{\concrete{ \{ } #1 \concrete{ \} }}
\newcommand{\equals}{\concrete{=}}
\newcommand{\comma}{\concrete{,}}
%%%%%%%%%%%%%%%%%%%%%%%%%%%%%
% The grammar figure
\begin{figure}[htb]
\begin{tabular}{lclr}
% Models
\esyntaxtopline{ \model }
               { \definitionList\ \systemEquation }
               { model }
\\
% Definition lists
\esyntaxtopline{ \definitionList }
               { \oneOrMoreBracketed{ \definition \concrete{;} } }
               { definition list }
\\
%% \esyntaxline   { \definition }
%%                { one definition }
%% \\
% Definitions
\esyntaxtopline{ \definition }
               { \ratedef }
               { rate definition }
\\
\esyntaxline   { \processdef }
               { process definition }
\\
\esyntaxline   { \namespace }
               { name space }
\\

% RateDefinitions
\esyntaxtopline{ \ratedef }
               { \rateId\ \equals\ \rateSet }
               { rate definition }
\\
\esyntaxtopline{ \rateSet }
               { \rateExp }
               { Single rate }
\\
\esyntaxline   { \concreteBraces{ \commaRateExps } }
               { set of rates }
\\
\esyntaxtopline{ \commaRateExps }
               { \rateExp\ \oneOrMoreBracketed{ \comma\ \rateExp } }
               { set of rates }
\\
% Process Definitions
\esyntaxtopline{ \processdef }
               { \optional{\concrete{ \# } } \processId
                 \concrete{ = } \process }
               { process definition }
\\
% Name space selection
\esyntaxtopline{ \namespace }
               { \namespaceId :: \concrete{ = }
                                 \concreteBraces{ \qualifiedNameSpaceId 
                                                  \oneOrMoreBracketed{ \comma\ \qualifiedNameSpaceId }
                                                }
               }
               { Name space selection}
\\
% Name space
\esyntaxtopline{ \namespace }
               { \namespaceId :: \concreteBraces{ \definitionList } }
               { Name space }
\\
% Components
\esyntaxtopline{ \process }
               { \qualifiedProcessId }
               { named process }
\\
\esyntaxline   { \prefix
                 \ \process
               }
               { Prefix components }
\\
\esyntaxline   { \concrete{ if }
                 \ \beCondition
                 \ \concrete{ then }
                 \ \prefix
                 \ \process
               }
               { conditional behaviour }
\\
\esyntaxline   { \process
                 \concrete{ + }
                 \process
               }
               { Choice }
\\
\esyntaxline   { \process
                 \concrete{ < }
                 \optional{ \actionList }
                 \concrete{ > }
                 \process
               }
               { Cooperation }
\\
\esyntaxline   { \process \concrete{ / }
                          \concreteBraces{ \actionList }
               }
               { Hiding }
\\
\esyntaxline   { \qualifiedProcessId 
                 \concrete{ [ }
                 \concrete{ \digits }
                 \concrete{ ] }
                 \optional{ \concrete{ [ }
                            \actionList
                            \concrete{ ] }
                          }
               }
               { process array }
\\
% Prefixes
\esyntaxtopline{ \prefix }
               { \concrete{ ( } \actionName
                 \concrete{,}
                 \rate
                 \concrete{ ) . }
               }
               { timed prefix }
\\
\esyntaxline   { \actionName 
                 \concrete{ . }
               }
               { immediate prefix }
\\

% conditions
\esyntaxtopline{ \beCondition }
               { \qualifiedRateId
                 \ \condOper
                 \ \rateExp
               }
               { behaviour condition }
\\

% condition operators
\esyntaxtopline{ \condOper }
               {      \concrete{ < }
                 \mid \concrete{ > }
                 \mid \concrete{ = }
                 \mid \concrete{ >= }
                 \mid \concrete{ <= }
                 \mid \concrete{ <> }
               }
               { condition operators }
\\
% Rates
\esyntaxtopline{ \rate }
               { \concrete{ infty } \mid \concrete{ \_ } }
               { passive }
\\
\esyntaxline   { \rateExp }
               { timed rate expression }
\\
\esyntaxline   { \forgivFloat
                 \concrete{ : immediate } 
               }
               { immediate }
\\
% Rate Expressions
\esyntaxtopline{ \rateExp }
               { \qualifiedRateId }
               { named rate }
\\
\esyntaxline   { \forgivFloat }
               { literal float }
\\
\esyntaxline   { \rateExp\ \rateOp\ \rateExp }
               { binary op expression }
\\
\esyntaxline   { \qualifiedProcessId }
               { process rate }
\\
\esyntaxline   { \concrete{ if }
                 \rateCond
                 \concrete{ then }
                 \rateExp 
                 \concrete{ else }
                 \rateExp
               }
               { conditional rate }
\\
               
% Rate Expression binary operators
\esyntaxtopline{ \rateOp }
               { \concrete{ + }
                 \mid
                 \concrete{ - }
                 \mid
                 \concrete{ / }
                 \mid
                 \concrete{ * }
               }
               { rate operators }
\\
% Rate conditions
\esyntaxtopline{ \rateCond }
               { \qualifiedProcessId }
               { Process present }
\\

% Forgiving Floats
\esyntaxtopline{ \forgivFloat }
               { \concrete{ \digits }
                 \optional{ \concrete{. \digits } }
               }
               { integer or float }
\\
% Action List
\esyntaxtopline{ \actionList }
               { \actionName \oneOrMoreBracketed{ \concrete{ , }
                                                  \actionName
                                                }
               }
               { comma action list }
\\

% Identifiers
\esyntaxtopline{ \actionName }
               { [\lowers]\zeroOrMore{[\lowers\ \uppers\ \digits]} }
               { lower identifier }
\\
\esyntaxtopline{ \rateId }
               { [\lowers]\zeroOrMore{[\lowers\ \uppers\ \digits]} }
               { lower identifier }
\\
\esyntaxtopline{ \processId }
               { [\uppers]\zeroOrMore{[\lowers\ \uppers\ \digits]} }
               { upper identifier }
\\
\esyntaxtopline{ \namespaceId }
               { [\uppers]\zeroOrMore{[\lowers\ \uppers\ \digits ]} }
               { all upper }
\\
\esyntaxtopline{ \qualifier }
               { \zeroOrMoreBracketed{ \namespaceId \concrete{::} } }
               { name qualifier }
\\
\esyntaxtopline{ \qualifiedRateId }
               { \qualifier\ \rateId }
               { qualified rate name }
\\
\esyntaxtopline{ \qualifiedProcessId }
               { \qualifier\ \processId }
               { qualified process name }
\\
\esyntaxtopline{ \qualifiedNameSpaceId }
               { \qualifier\ \namespaceId }
               { qualified process name }
\\
% Finallly the system description
\esyntaxtopline{ \systemEquation }
               { \process }
               { main system }
\\
\end{tabular}
\caption{
\label{figure:syntax:srmcfiles}
The concrete syntax for srmc model files
}
\end{figure}
