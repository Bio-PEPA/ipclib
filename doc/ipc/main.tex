% We must define this as the options file will output tex
% containing this comamnd.
\newcommand{\showcommandline}[1]{
\begin{commandline}
\$ ipc #1
\end{commandline}
}

\newcommand{\commandNameIpcSmc}{\ipc}

\clearpage

\section{Usage}

\subsection{Trivial Invocation Options}
These options only apply to the trivial
invocations of \commandNameIpcSmc.

\begin{description}
\item[-v, \ipcflag{version} ]
\indexipcflag{ version }%
The command-line:

\showcommandline{\ipcflag{version}}
will print the version of the \commandNameIpcSmc
compiler and exit.

\end{description}

\begin{description}
\item[-h, \ipcflag{help} ]
\indexipcflag{ help }%
The command-line

\showcommandline{\ipcflag{ help}}

will print an option and usage summary and exit.

\end{description}


\subsection{Output Re-directing Options}
By default the \commandNameIpcSmc\ command
will generate output in the same directory and
with a file name based on the input source name
The options in this section allow the user to
redirect the output from the
\commandNameIpcSmc\ compiler

\begin{description}
\item[\ipcflag{mod-file} FILE]
Sets the output \texttt{.mod} file which will be
the input to the run of \hydra

\end{description}

\begin{description}
\item[\ipcflag{output} FILE]
A generic output flag, this works regardless of the kind
output that \commandNameIpcSmc\ is set to produce. For example
instead of the default \hydra\ the compiler may have
been set to produce a PRISM model file.
The \ipcflag{output} can be used to set the
output file name

\end{description}

\begin{description}
\item[\ipcflag{stdout} ]
When debugging the compiler it can often be useful
for the output to be redirected to the terminal for
immediate inspection by the programmer.
The \ipcflag{stdout} sets the output file to be
the standard out. This may also prove useful for
piping the output into further processing tools.

\end{description}

\begin{description}
\item[\ipcflag{graph-output} GRAPHKIND]
When outputting a graph we wish to be able to select
what kind of output we want.
For this use the \ipcflag{graph-output} flag

\end{description}

\begin{description}
\item[\ipcflag{line-width} WIDTH]
When drawing a line graph we can select the width of a line
using the \ipcflag{line-width} flag

\end{description}


\subsection{Detecting Errors}
These options control the way that
static analysis is performed.

\begin{description}
\item[\ipcflag{staunch} ]
The flag \ipcflag{staunch} allows the compiler
to ignore warnings.
By default this flag is off and when \commandNameIpcSmc\ performs
any static-analys is over the input \pepa\ model.
The default behaviour is to treat warnings as errors so
if there are any warnings it will cause the compiler
to cancel compilation. This behaviour can be suppressed with
the \ipcflag{staunch} flag. This will cause the warnings
to still be emitted but compilation will proceed anyway.

\end{description}

\begin{description}
\item[\ipcflag{no-static-analysis} ]
The flag \ipcflag{no-static-analysis} causes
\commandNameIpcSmc\ to avoid performing the static analysis over
the \pepa\ model. It will therefore produce no
warnings or errors. Because of this compilation may fail
mysteriously and hence the user is advised only to use this
flag if they know exactly what they are doing and expect
their model to fail static-analysis for some reason but
wish to proceed to compilation anyway.

\end{description}

\begin{description}
\item[\ipcflag{allow-self-loops} ]
The flag \ipcflag{allow-self-loops} does exactly as it
is named. We allow a model containing self-loops.
This can allow the correct calculation of apparent
rates, but note that the self-looping activities
will be dropped (hence your throughput may be wrong.

\end{description}

\begin{description}
\item[\ipcflag{allow-deadlocks} ]
The flag \ipcflag{allow-deadlocks} does exactly as it
is named. We allow a model which has deadlocked states.
Mostly this is only useful for transient analysis so for
transient analysis it is the default. It is however also
possible to use this (usefully) for passage-time/end
analyses in which there is a single source state

\end{description}


\subsection{Performance Analysis Kind}
In this section the options for specifying the
different kinds of performance analysis are discussed.

\begin{description}
\item[\ipcflag{average-response} ]
Use this option to calculate the average response time
The response-time will be the time that the probe is
in the running state. This probe may be specified via
the \ipcflag{probe} argument or implicitly from a set
of source actions to a set of target actions.

\end{description}

\begin{description}
\item[\ipcflag{passage} ]
Specifies that we should perform a passage-time measurement.
This is the default. It also means that the passage-probe
is added to the model. This probe is in addition to the
master probe. It is used as the start condition of passage.
The passage-probe has two states an 'on' and 'off' state.
It will be in the 'on' state for exactly one state after the
master-probe has switched from the 'Stopped' to the 'Running'
state and 'off' otherwise.

\end{description}

\begin{description}
\item[\ipcflag{pdf} ]
When specifying a passage-time measurement state that the
probability density function should be calculated

\end{description}

\begin{description}
\item[\ipcflag{cdf} ]
When specifying a passage-time measurment state that
the cummulative distribution function of the passage
should be calculated.
This is the default however should you desire both the cdf
and the pdf, then --cdf --pdf is what you want.

\end{description}

\begin{description}
\item[\ipcflag{passage-end} ]
Specifies a special kind of passage-time calculation
in which we compare the likelihood of completing a passage
by the different stop-actions at each time.
For example we may say that if we complete a request by
time t then we have a ninety-percent chance that we
completed the request via a cache but at time 2t there is a
fifty percent chance that any request completed by then has
been served by the cache

\end{description}

\begin{description}
\item[\ipcflag{no-normalise} ]
A passage-end measurement is usually normalised against
the probability of completing the passage at all.
To suppress this and produce non-normalised results use
the \ipcflag{no-normalise} flag.
This will mean for example that you will get cdfs which
do not climb to one.

\end{description}

\begin{description}
\item[\ipcflag{steady} ]
Specifies that we should perform a steady-state analysis.
By default this will measure the probability that the probe
is in the 'Running' state.

\end{description}

\begin{description}
\item[\ipcflag{transient} ]
Specifies that we should perform a transient analysis

\end{description}

\begin{description}
\item[\ipcflag{count-measure} ACTIONS]
Specifies that we should perform a 'count' measure.
This option takes as argument the (comma separated) list of
action name we are expected to count.
This will cause hydra to return the average rate that the model
performs any of the specified actions.
Note that the specified action can be a communication label sent
by a probe specified by the user.

\end{description}

\begin{description}
\item[\ipcflag{state-measure} C-Expression]
When performing a steady-state measurement rather than
specifying the state of interest via a sequence of
actions (ie, with a probe) sometimes it is desirable to
specify it with reference to the actual states of the
components within the model. The argument to this flag
is the a state expression usually something like: 
$P3 > 0 \&\& P2 == 0$.

\end{description}

\begin{description}
\item[\ipcflag{no-measurement} ]
Suppresses the output of any measurement specification
This is generally useful if the user wishes to hand-modify
the output \texttt{.mod} file before running hydra over it.
% Therefore this option is generally run in tandem with the
% --no-run-hydra option.

\end{description}


\subsection{Probe Specification Options}
Measurement of models is performed using automatically
generated process algebra components called
\emph{stochastic probes}
This section describes the options which control
the performance measure specification probes which
are added to the model.
% For more information on this please see section ...

\begin{description}
\item[-p, \ipcflag{probe} PROBESPEC]
The flag \ipcflag{ probe } with the shortened version
\texttt{p} is used to give a full probe specification.
% In the probe specification language given in section ??

\end{description}

\begin{description}
\item[\ipcflag{no-master} ]
The flag \ipcflag{ no-master } is used to specify that
no master probe % as described in section ...
should be automatically added to the model.

\end{description}

\begin{description}
\item[-s, \ipcflag{source} ACTIONS]
The flags \ipcflag{ source } and \ipcflag{ target }
with the short versions \texttt{s} and \texttt{t}
respectively, set the state switching actions used
in the master probe. If other probes are added using the
\ipcflag{ probe } option then they may perform immediate
communication actions which are specified in the source and
target action list.
Both the \ipcflag{ source } and \ipcflag{ target } flags
accept as argument a comma separated list of action names.

\end{description}

\begin{description}
\item[-t, \ipcflag{target} ACTIONS]
see \texttt{--source}
\end{description}

\begin{description}
\item[\ipcflag{source-cond} CONDITION]
The flags \ipcflag{source-cond} and \ipcflag{target-cond}
allow the specification of a passage-time measurement 
with respect to state conditions rather than action
observations.

\end{description}

\begin{description}
\item[\ipcflag{target-cond} CONDITION]
see \texttt{--source-cond}
\end{description}


\subsection{Aggregation}
The options in this section control the aggregation
of components.
Currently in \commandNameIpcSmc\ a component may
only be aggregated if the user explicitly writes it
as a component array of the form $P[N]$.

\begin{description}
\item[\ipcflag{aggregate} ]
Tells \commandNameIpcSmc\ to aggregate process arrays

\end{description}

\begin{description}
\item[\ipcflag{no-aggregate} ]
Tells \commandNameIpcSmc\ \textbf{not} to aggregate
process arrays. Therefore $P[3][a]$ will be translated
into the form $P <a> P <a> P$.

\end{description}

\begin{description}
\item[\ipcflag{limit} SIZE]
Tells the state space generator to quit after reaching a
given limit in the state space size

\end{description}


\subsection{Running Hydra}
This section details options used for running
the hydra tool after processing of the model by \ipc.

\begin{description}
\item[\ipcflag{run-hydra} ]
The flag \ipcflag{run-hydra} causes \commandNameIpcSmc\ to
automatically run the hydra tool on the produced
\texttt{.mod} file.
This option has been deprecated please use the
\ipcflag{hydra-stage} option described below.

\end{description}

\begin{description}
\item[\ipcflag{hydra} PATH]
If the hydra tool is not installed in a standard
location the path to the \hydra\ executable can
be given as an argument to the \ipcflag{hydra} option.

\end{description}

\begin{description}
\item[\ipcflag{hydra-stage} STAGE]
Choose at which stage we should stop using ipc
and switch over to hydra.
 This option replaces the older \ipcflag{run-hydra} option.
There are currently two stages which may be specified:
\begin{itemize}
\item 'mod' this outputs a hydra model file in the original
format. The full state space is not computed by \ipc.
\item 'flat-mod' here the hydra model file contains a full
state space and as such does not suffer from a problem of
very large rate expressions.
\end{itemize}

\end{description}


\subsection{Prism Model Files}
At some point we wish to be able to generate prism
model files from Pepa model descriptions.
All of the options described in this section should be
considered experimental and not for general use.

\begin{description}
\item[\ipcflag{prism} STAGE]
\ipcflag{prism} this is an experimental option to produce
a prism model.
When using this one specifies a stage to which we with to
to compile the pepa model to.
Currently there are two supported options "trans" and
"model". The 'model' option should be considered not to be
working since it is in a very early stage of development
The 'trans' stage will compile the model down to an
explicit state space which prism can then import.

\end{description}

\begin{description}
\item[\ipcflag{prism-file} FILE]
The \ipcflag{prism-file} option only has an effect if the
\texttt{--prism} flag is set. It redirects the output
prism model to the specified file.

\end{description}

\begin{description}
\item[\ipcflag{prism-options} OPTIONS]
The \ipcflag{prism-options} flag has the effect to
append the provided string on to the end of the
the prism command

\end{description}


\subsection{Dizzy Model Files}
At some point we wish to be able to generate dizzy
model files from Pepa model descriptions.
The options in this section describe the control of
the \pepa\ $\rightarrow$ dizzy translation.

\begin{description}
\item[\ipcflag{dizzy} ]
\ipcflag{dizzy} this is an experimental option to produce
a dizzy model. This should not be considered working

\end{description}

\begin{description}
\item[\ipcflag{dizzy-file} FILE]
The \ipcflag{dizzy-file} option only has an effect if the
\texttt{--dizzy} flag is set. It redirects the output
dizzy model to the specified file.

\end{description}


\subsection{Extra Output}
The model undergoes various transformations and
augmentations on its way to being compiled.
Sometimes it is helpful to see these intermediate
stages. The options in this section allow the user
to specify that \commandNameIpcSmc\ should
include a particular intermediate model in the output,
generally inside comments of the output file.

\begin{description}
\item[\ipcflag{show-simplified} ]
Specifies that \commandNameIpcSmc\ should show the
simplified model. This is a model without any of the
syntactic sugar which the user may use for convenience
but are unnecessary for the compilation procedure

\end{description}

\begin{description}
\item[\ipcflag{show-probed} ]
Specifies that the model with the measurement probe
components added and then simplified is shown.

\end{description}


\subsection{Logging Options}
The options in this section affect what, if anything
is logged and how it is logged

\begin{description}
\item[\ipcflag{show-log} ]
Specifies that the logging information should be shown to stdout

\end{description}

\begin{description}
\item[\ipcflag{log} categories]
The flag \ipcflag{log} specifies that a log should be kept
Each log entry is associated with a name and the argument
to the \ipcflag{log} flag is a name specifying that that
entry should be output to the log file.
The name 'all' specifies that all log entries
should be recorded in the log-file.
Additionally the 'user' name outputs to the log items
deemed relevant to the user while 'developer' is a set
of log items relevant for the developer.

\end{description}


\subsection{Miscellaneous Options}
This section describes options which do not fit
under any of the previous sub-sections.

\begin{description}
\item[\ipcflag{steady-mean} ]
No manual entry but the usage information states:
use the 'mean' estimator in a steady-state measure
\end{description}

\begin{description}
\item[\ipcflag{steady-variance} ]
No manual entry but the usage information states:
use the 'variance' estimator in a steady-state measure
\end{description}

\begin{description}
\item[\ipcflag{steady-stddev} ]
No manual entry but the usage information states:
use the 'stddev' estimator in a steady-state measure
\end{description}

\begin{description}
\item[\ipcflag{steady-distrib} ]
No manual entry but the usage information states:
use the 'distribution' estimator in a steady-state measure
\end{description}

\begin{description}
\item[\ipcflag{start-time} TIME]
No manual entry but the usage information states:
specify a time at which to start a performance measure eg passage-time
\end{description}

\begin{description}
\item[\ipcflag{stop-time} TIME]
No manual entry but the usage information states:
specify a time at which to stop a performance measure eg passage-time
\end{description}

\begin{description}
\item[\ipcflag{time-step} TIME]
No manual entry but the usage information states:
specify a the time steps for a performance measurement
\end{description}

\begin{description}
\item[\ipcflag{solver} SOLVER]
No manual entry but the usage information states:
specify which solution method to use/specify to hydra
\end{description}

\begin{description}
\item[\ipcflag{rate} DOUBLE]
No manual entry but the usage information states:
Override/specify a rate value on the command-line
\end{description}

\begin{description}
\item[\ipcflag{rename-proc} P=s]
No manual entry but the usage information states:
cause a renaming on the given process within the model
\end{description}

\begin{description}
\item[\ipcflag{rename-rate} r=s]
No manual entry but the usage information states:
cause a renaming on the given rate within the model
\end{description}

\begin{description}
\item[\ipcflag{transform-rule} RULE]
Provide a transformation rule with which
to automatically transform the PEPA model

\end{description}

\begin{description}
\item[\ipcflag{prioritise} ACTIONS]
No manual entry but the usage information states:
increase the priority of the given actions
\end{description}

\begin{description}
\item[\ipcflag{dot-file} ]
Produce a .dot file of the state space and run the dot program over it
\end{description}

\begin{description}
\item[\ipcflag{no-reduce-vanishing} ]
When producing a state space for external output
do not remove the vanishing states

\end{description}

\begin{description}
\item[\ipcflag{no-reduce-rate-exps} ]
No manual entry but the usage information states:
do not reduce the rate expressions
\end{description}

\begin{description}
\item[\ipcflag{hide-non-coop} ]
No manual entry but the usage information states:
hide any activities which a component performs but does not cooperate over
\end{description}

\begin{description}
\item[\ipcflag{process-num} NUM]
No manual entry but the usage information states:
provide a process number which is used to select rates and processes
\end{description}

\begin{description}
\item[\ipcflag{fsp} ]
The \ipcflag{fsp} flag is an experimental option to produce
an LTSA model. This should not be considered working

\end{description}

\begin{description}
\item[\ipcflag{states-size} ]
The flag \ipcflag{states-size} informs \commandNameIpcSmc\ to
print out the size of the state space of the given model

\end{description}

\begin{description}
\item[\ipcflag{estimate-size} ]
The flag \ipcflag{estimate-size} asks \commandNameIpcSmc\ to simply
estimate the final state space size of the input model

\end{description}

\begin{description}
\item[\ipcflag{compare-pepato} ]
The flag \ipcflag{compare-pepato} informs \commandNameIpcSmc\ to
produce a steady-state analysis and run pepato over
the model to also produce a steady-state analysis and
compare the two reports.

\end{description}

\begin{description}
\item[\ipcflag{experimental} ]
The flag \ipcflag{expermental} is for developers only
This is a generic flag meaning:
"enable a new approach/version of .."
Generally only to be used by ipc developers.
So for example a new approach to state space generation,
rather than inventing a new flag
"--use-new-state-space-gen" we can just test for this.
If later we decide that we do actually wish to have both
*then* we can invent two flags for it.

\end{description}





\subsection{Using \ipc\ to evaluate a single model}
In this section we will learn how to invoke \ipc\ to peform
various analyses over simple models. 
%% The concrete syntax should be give

As a simple example consider the \pepa\ model shown here in
the concrete syntax of \ipc.
\verbatiminput{models/tiny.pepa}

\subsection{Basic Invocation}
Invoking the compiler to give basic information can be done by
specifying either the \ipcflag{version} or \ipcflag{help} flags.
\begin{verbatim}
ipc --version
\end{verbatim}

\begin{verbatim}
ipc --help
\end{verbatim}

Many of the flags and options have a corresponding one letter alias,
for example the version can be printed with the command:
\begin{verbatim}
ipc -v
\end{verbatim}
For the remainder of this manual we will use the long versions for
clarity.  The short versions can be found by reading the output of the
help flag.

Beyond the trivial invocations involving version and help information,
\ipc\ must be invoked with a \pepa\ model argument.
The general form is
\begin{verbatim}
ipc [flags and options] tiny.pepa
\end{verbatim}

The simplest command-line for our \texttt{tiny.pepa} file would be:
\begin{verbatim}
ipc tiny.pepa
\end{verbatim}

Such a simple command will display a steady state probability distribution
of all the states in your model. In the future we hope to make this a little
more useful by providing names of components, utilisation and throughput of
activities.
Generally of course the point of solving the model is to make some
measurement of the model. The following section
\ref{specsourceandtarget} begins with the simplest way in which
to specify a measurement.

%   [
%     Option "v"     ["verbose"]     (NoArg CliVerbose)        
%     "logfile output to STDOUT"

%   , Option "V"     ["version"]     (NoArg CliVersion)        
%     "show version info"

%   , Option "h"     ["help"]        (NoArg CliHelp)
%     "show options and documentation"

\subsubsection{Specifying source and target options}
\label{specsourceandtarget}
%   , Option "s"     ["source"]      (ReqArg startActions      "STARTACTIONS" )   
%     "start actions for stochastic probe"

%   , Option "t"     ["target"]      (ReqArg stopActions       "STOPACTIONS" )    
%     "stop actions for stochastic probe"

The \ipc/Hydra tool chain can be used to calculate three major types of
performance measure:
\begin{enumerate}
\item Passage time quantiles and distributions
\item Transient measures
\item Steady state measures
\end{enumerate}

There are two ways in which to specify the states in which the
measurement is concerned. 
In this section simple measurements which can be done by specifying
sets of source and target actions are considered.
Section \ref{section:probespecifications} is concerned with more
complex measurements.

As the names suggest, the source activity is used
to start the measurement and the target activity is used to end the measurement.

Thus if we were interested in computing the passage from an occurrence
of the \texttt{start} activity to the occurrence of the \texttt{stop}
activity by either of the copies of the process in the tiny model then
we would use the following command
\indexipcflag{source}%
\indexipcflag{target}%
\begin{verbatim}
ipc --source start --target stop tiny.pepa
\end{verbatim}

This will generate two comma separated files:
\texttt{tiny\_cdf.csv} and \texttt{tiny\_pdf.csv}.
These are the cummulative distribution function and the probability
density function of the specified passage respectively.

With each of these arguments more than one action may be given as a 
comma separated list of actions.
For example one might give a command such as:
\begin{verbatim}
ipc --source halfFull,halfEmpty --target full,empty optimism.pepa
\end{verbatim}

% Since \hydra\ will more often than not be invoked on the output
% \texttt{.mod} file one can specify the \ipcflag{run-hydra} flag.
% By default the kind of measurement is a passage-time measurement
% and \ipc\ knows this and will run the corresponding
% \texttt{hydra-uniform} command after the initial run of
% \hydra\ itself. So by issuing the command:
% \begin{verbatim}
% ipc tiny.pepa --source start --target stop --run-hydra
% \end{verbatim}
% You should produce the file: \texttt{tiny.PT\_RESULTS}.
% The \gnuplot\ program can then be used to produce a graph of
% the results such as the one in Figure
% \ref{figure:tiny-passage-time-results}.
% This graph plots the cummulative probability of completion
% of the passage against time. That is the probability that
% $t$ seconds after observing a $start$ activity a $stop$ activity
% is observed.

\begin{figure}{htb}
% \includegraphics{models/tiny_cdf.pdf}
\caption{
\label{figure:tiny-passage-time-results}
Shows the cummulative probability of completion against time.
}
\end{figure}

\begin{figure}{htb}
%\includegraphics{models/tiny_pdf.pdf}
\caption{
\label{figure:tiny-passage-time-pdf-results}
Shows the probability density of completion against time.
}
\end{figure}


\subsection{Changing the analysis type}
We can specify the three different kinds of analysis with the
three flags:
\begin{itemize}
\item \ipcflag{steady}
\item \ipcflag{transient}
\item \ipcflag{passage}
\item \ipcflag{no-measurement}
\end{itemize}

A passage-time measure is the default unless the user does not
specify either a measurement probe or a set of source/target actions.
In this case \ipc\ defaults to no measurement.

\TODO{Explain what each of the measurement kinds mean and in
particular for the given example what it would compute}.
% For steady-state then this would compute the probability that
% we are in a state in which we have observed a start action without
% having observed a stop action.

\subsection{Re-directing the output}
\label{redirectoutputsection}
% By default \ipc\ outputs logging information.  The default file to
% output the log information to is based on the name of the PEPA model
% file.  With the command line
% \begin{verbatim}
% ipc  --source start --target stop tiny.pepa
% \end{verbatim}
% the logging information will be placed in the file \texttt{tiny.log}.
% This can be changed with the \ipcflag{logfile} option such as
% \begin{verbatim}
% ipc  --source start --target stop --logfile ipc.log tiny.pepa
% \end{verbatim}

For all output files one can redirect it by giving the \ipcflag{output}
flag. This will redirect a file of the same suffix. So for example
if you give

\begin{verbatim}
ipc -s a -t b --output picture_cdf.csv simple.pepa
\end{verbatim}

Then the cdf comma separated value file will be placed in
\texttt{picture\_cdf.csv} rather than \texttt{simple\_cdf.csv}.

% The output model used as input to the \hydra\ tool will by
% default be placed in the file \texttt{tiny.mod}, but this can be
% overridden with the command line option 
% \ipcflag{mod-file} as in:
% \begin{verbatim}
% ipc --source start --target stop --mod-file ipc.mod tiny.pepa
% \end{verbatim}

\ifDeveloperVersion
To aid debugging it's often useful to dump the output to standard
out for inspection. The \ipcflag{stdout} flag redirects the
output to the terminal.
\else
\fi

% As a sometimes more convenient method of redirecting output the user
% can specify an output directory. So we can run the following commands:
% \indexipcflag{--outputdir}
% \begin{verbatim}
%    mkdir output
% ipc --source start --target stop --outputdir output tiny.pepa
% \end{verbatim}
% This will have the effect that all of the files generated by this run
% of \ipc\ will be placed in the \texttt{output} directory,
% including the \texttt{.mod} the \texttt{.log} files and also other
% generated files which are explained in the sections that follow.
% In particular see Sections \ref{Condoroutputsection} and
% \ref{processdatabaseoutputsection}.

Note that as mentioned above we will be giving command line options in
their long forms, but most of them have short forms. The
\texttt{--source} and \texttt{--target} options have the short forms
\texttt{-s} and \texttt{-t} respectively. Because start and stop
actions must be given for almost all runs of \ipc\ we will use the
short options to reduce the length of command lines where appropriate
in the remainder of this manual.

% Todo: what exactly does this do? and does it suppress the later
% modelling, I think it does.
%   , Option "P"     ["pepaoutput"]  (ReqArg CliPepaOutput     "FILE" )           
%     "produce PEPA with probe to FILE"

%\subsubsection{Using legacy active rate calculation}
% Allan look in the code and find out exactly what this does!!
%   , Option "W"     ["pwbcompat"]   (NoArg CliPwbCompat)      
%     "use legacy active rate calculation"


\subsection{Specifying start and stop times}
%   , Option ""      ["starttime"]   (ReqArg startTime         "STARTTIME" )
%     "Sets the start time"

%   , Option ""      ["stoptime"]    (ReqArg stopTime          "STOPTIME"  )
%     "Sets the stop time"

%   , Option ""      ["timestep"]    (ReqArg stepTime          "TIMESTEP"  )
%     "Sets the value of the time step"
% Via Hydra, \ipc\ computes performance measures such as probability
% density functions (PDFs) and cumulative distribution functions (CDFs).
% Together the PDF and CDF give a complete description of the
% probability distribution of a random variable.  Both are evaluated
% against increasing time in order to produce function plots.  \ipc\ has
% default values for the start time and stop time for the plot,
% as well as the time step which determines how many evaluations of these
% functions Hydra is to do.  The command-line options 
% \ipcflag{start-time}, \ipcflag{stop-time} and \ipcflag{time-step}
% allow us to override these defaults.

To specify a start time of 50 seconds and an end time of 100 seconds
with a timestep of 10 seconds we would pass the following command-line
options to \ipc.
\begin{verbatim}
ipc --start-time 50 --stop-time 100 --time-step 10 ...
\end{verbatim}

\section{Probe Specifications}
\label{section:probespecifications}

\ifDeveloperVersion
\section{Hydra Compilation Details}
This section details the transformations that the input pepa model goes
through in the translation to a \hydra\ model.
% The other compilation targets will be discussed in the following sections.
We begin by listing the stages and then each stage will be discussed in
greater detail.

\begin{itemize}
\item The input \pepa\ model is parsed.
\item The syntactic sugar forms are removed: these are such constructs as:
   \begin{itemize}
      \item Parallel Definitions eg $P = Q <> R$.
      \item Non-aliased sub-components, eg $(a, r).P + (b, r).Q$ is turned
            into $P1 + Q1$ with the appropriate definitions added.
   \end{itemize}
\item Qualify the model. Briefly this means that each defined process is
used exactly once from the main cooperation.
\item Remove the hiding operator
\end{itemize}

At this point the model is still in \pepa\ format and could be output
suitably for other \pepa\ related tools. 

\begin{itemize}
\item This model is then translated into a structured form of the
\hydra\ transitions. Essentially this is sets of dnam transitions which may
cooperate with each other.
\item The structured sets of \hydra\ transitions are flattened into one list.
At this point the apparent rates are calculated.
\item The \hydra\ model is formatted appropriately for the
\hydra\cite{pepahydra}
Markov chain analyser.
\end{itemize}

\subsection{Removal of Syntactic Sugar}
As we will see later \pepa\ models of a specific form are more convenient to
convert into a model suitable for \hydra.
The specific form of a \pepa\ model is a normal form, the grammar for
this form of \pepa\ model is given in Figure
\ref{figure:pepaNormalForm:grammar}.
The interesting notions are the following:
\begin{itemize}
\item There are no parallel definitions
eg( $P \rmdef Q \sync{} R$ )
These are removed by substituting their definitions into the main system
equation.
\item There are no alias definitions,
eg( $P \rmdef Q$). Again these are removed by substitution.
\item All component sums have only identifiers as their left and right choices.
\item Prefix operations only have identifiers as their successor components.
\end{itemize}

\begin{figure}
\begin{eqnarray*}
def & := & P \rmdef sequential ;    \\
sequential & :=   & (a, r) . P      \\
           & \mid & P + Q           \\
parallel   & :=   & P               \\
           & \mid & P < actions > Q \\
           & \mid & P / { actions } \\
           & \mid & P[N][actions]   \\
model      & :=   & def^+ parallel  \\
\end{eqnarray*}
\caption{
\label{figure:pepaNormalForm:grammar}
The grammar for a pepa model in normal form.
}
\end{figure}

\subsection{Qualifying the Model}

\adcComment{Definitely rewrite this better.}
The model in normal form however is still not ready to be converted into a
\hydra\ model. We must distinguish between each invocation of a given defined
process. The transformation for this must be deep, in that it must see through
definitions otherwise one process may become another one and hence violate
the uniqueness property that we require.

The transformation provided adds a suffix number to each process name.
Figure
\ref{figure:qualifyModel}
gives an example of qualifying a very simple \pepa\ model.
We start with the main composition and the list of original definitions
are used as a dictionary but are not included in the final model.
The transformation then must return a new main parallel composition
plus a new list of process definitions.

\begin{figure}
\begin{eqnarray*}
P & \rmdef & (a, r) . Q \\
Q & \rmdef & (b, r) . P \\
  & & P \sync{a} P
\end{eqnarray*}
\begin{eqnarray*}
P_1 & \rmdef & (a, r) . Q_1 \\
Q_1 & \rmdef & (b, r) . P_1 \\
P_2 & \rmdef & (a, r) . Q_2 \\
Q_2 & \rmdef & (b, r) . P_2 \\
  & & P_1 \sync{a} P_2
\end{eqnarray*}
\caption{
\label{figure:qualifyModel}
An example model and the qualified version.
}
\end{figure}

The first part of the algorithm requires us to qualify a sequential
process. We turn a given process identifier into a list of new definitions.
We assume that $lookup(P)$ will return the sequential component which
the process identifier $P$ is bound to in the original list of process
definitions (which will ultimately be discarded).

In order to do this we define two mutually recursive functions.
Since most components will be cyclic we must end the recursion,
we do this by maintaining a list of 'seen' process identifiers.
Whenever we attempt to redefine a process in the 'seen' list then
we return the empty set of new definitions.
The $T\_def$ function checks if we have already seen the given
process identifier and if not looks it up the initial list of
process definitions. The $T$ function takes in a process name
to define and a sequential component to qualify.
Notice that this algorithm is trivial because we already know
that the model is in normal form.

\begin{verbatim}
T(i, seen, P, (a, r).Q) = [ P_i = (a, r) . Q_i ] union
                          T_def (i, seen, Q)
T(i, seen, P, Q + R)    = ( [ P_i = Q_i + R_i ] union
                            T_def(i, seen, Q) union
                            T_def(i, seen, R)

T_def = (i, seen, P)    = if P is in seen then []
                          else T(i, P:seen, P, lookup(P))
\end{verbatim}


The second part of the algorithm proceeds via recursion through the main
system component. This simply updates a counter upon each process encountered.
Because we know we have already removed parallel definitions and aliases each
encountered process can be assumed to be a sequentially defined component.
We must return three results; the list of newly defined components, the updated
suffix number, and the qualified parallel process.

\begin{verbatim}
T(i, P)             = ( i+1, T_def(i, [], P), P_i )
T(i, r <acts> s)    = let (j, defs_r, r_i) = T(i, r)
                          (k, defs_s, s_i) = T(j, s)
                      (k, defs_r union defs_s, r_i <acts> s_i)
T(i, p/{actions})   = let (j, defs, p_i) = T(i, p)
                          (j, defs, p_i/{actions}
T(i, P[N][actions]) = let (j, defs, p_i) = T_def(i, [], p)
                          (j, defs, p_i/{actions}
\end{verbatim}

\subsection{Removal of the Hiding Operator}
The hiding operator can now be trivially removed.
This is because the model is in qualified form.
The hiding operator can be removed simply by renaming the hidden actions.
This works because each occurrence of a process is unique therefore
its definition is used only once and each action that it may perform
is either hidden or not hidden.
Figure \ref{figure:removeHiding:Model}
gives an example qualified model and the same model with the hiding operator
removed. Notice that the processes $P_1$ and $P_2$ may still cooperate over
the $a$ action, though it is renamed to $a_1$.

\begin{figure}
\begin{eqnarray*}
P_1 & \rmdef & (a, r) . Q_1 \\
Q_1 & \rmdef & (b, r) . P_1 \\
P_2 & \rmdef & (a, r) . Q_2 \\
Q_2 & \rmdef & (b, r) . P_2 \\
P_3 & \rmdef & (a, r) . Q_3 \\
Q_3 & \rmdef & (b, r) . P_3 \\
  & & (P_1 \sync{a} P_2)/\{a\} \sync{} P_3
\end{eqnarray*}
\begin{eqnarray*}
P_1 & \rmdef & (a_1, r) . Q_1 \\
Q_1 & \rmdef & (b, r) . P_1 \\
P_2 & \rmdef & (a_1, r) . Q_2 \\
Q_2 & \rmdef & (b, r) . P_2 \\
P_3 & \rmdef & (a, r) . Q_3 \\
Q_3 & \rmdef & (b, r) . P_3 \\
  & & (P_1 \sync{a_1} P_2) \sync{} P_3
\end{eqnarray*}
\caption{
\label{figure:removeHiding:Model}
An example qualified model showing the removal of the hiding operator.
}
\end{figure}

\subsection{Structured \hydra\ Transitions}
Because of the previous set of transformations the \pepa\ models which
reach this stage of compilation are in a very strict form.
Each sequential component can be easily tranformed into a list of transitions
which it may perform. Where a transition consists of a rate and the
pre-conditions for the transition to be active, and the post-conditions
or effects on the model's state that the transition has.
The pre-conditions at this
stage is simply that there be at least one of the named process.


Sequential components then are turned into a list of transitions for the
$P_1$ component of the example given in Figure
\ref{figure:removeHiding:Model}
we can make the list of transitions:

\begin{verbatim}
{ kind = a
  rate = r
  preconditions  = P_1 > 0
  postconditions = P_1 = P_1 - 1
                   Q_1 = Q_1 + 1   
}
{ kind = b
  rate = r
  preconditions  = Q_1 > 0
  postconditions = Q_1 = Q_1 - 1
                   P_1 = P_1 + 1
}
\end{verbatim}

These sets are of transitions are stored within the original models structure
for the main system equation. So that we have sets of transitions cooperating
over action names.

\subsection{Flattened List of Transitions}
The structured sets of transitions are combined together into single list
of transitions. Where the transitions are not cooperated over this is a simple
union. Where the transitions cooperate then two transitions are combined into a
single one. The apparent rate of the single transition is calculated at this
stage.

% \section{The Compilation of Process Arrays}
\else
\fi


\section{Attaching Performance Measurement Probes}
