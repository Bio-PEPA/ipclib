\documentclass[10pt,a4paper]{article}
\usepackage[usenames]{color}

\usepackage{amssymb,amsbsy,verbatim,fancybox, ../common/pepa}
\usepackage{graphicx}
%% \usepackage{lhs2TeX}

\newcommand{\adcComment}[1]{\textbf{#1}}
\newcommand{\adcCommentl}[1]{\textbf{#1}\\}
\newcommand{\rewordthis}[1]{\textsl{#1}\\}
\newcommand{\adcTodo}{\adcComment{TODO}}

\newcommand{\ipc}{\textsf{ipc}}
\newcommand{\srmc}{\textsf{srmc}}
\newcommand{\smc}{\textsf{smc}}
\newcommand{\hydra}{\textrm{Hydra}}
\newcommand{\Condor}{\textrm{Condor}}
\newcommand{\hydrus}{\textsf{hydrus}}
\newcommand{\collate}{\textsf{collate}}
\newcommand{\puffball}{\textsf{puffball}}
\newcommand{\gnuplot}{\textsf{gnuplot}}
\newcommand{\pepa}{\textrm{PEPA}}
\newcommand{\sectref}[1]{Sect.~\ref{#1}}
\newcommand{\figref}[1]{Fig.~\ref{#1}}

% The if commands for distinguishing between the developer and user manuals
\newif\ifDeveloperVersion

\newcommand{\indexipcflag}[1]{\index{\ipc command-line flag!\texttt{#1}}}
\newcommand{\ipcflag}[1]{\indexipcflag{#1}\texttt{--\ignorespaces#1}}
\newcommand{\solver}[1]{\index{\ipc command-line flag!\texttt{--solver}!\texttt{#1}}\texttt{#1}}
\newcommand{\figSpec}[2]{
  \begin{figure}[t]
    \begin{center}
      \includegraphics[width=0.6\textwidth]{figs/#1}
      \caption[ ] {\label{fig:#1} #2}
    \end{center}
  \end{figure}
}

%%%%%%%%%%%%%%%%%%%%%%%%%%%%%%%%%%%%%%%%%%%%%%%%%%%%
%%  Url definitions
%%%%%%%%%%%%%%%%%%%%%%%%%%%%%%%%%%%%%%%%%%%%%%%%%%%%

% For now just until I work out how to check
% whether or not we are using hevea

\newif\ifusingHevea
\usingHeveafalse


\ifusingHevea
\else
\newcommand{\ahrefurl}[1]{#1}
\fi


\newcommand{\ipcstartpage}{
\ahrefurl{\url{http://homepages.inf.ed.ac.uk/s9810217/software/ipc}}}
\newcommand{\usingpuffballurl}{
\ahrefurl{\url{http://homepages.inf.ed.ac.uk/s9810217/software/puffball/usingpuffball.html}}\xspace}
\newcommand{\ipcsrctarballurl}{
\ahrefurl{\url{http://homepages.inf.ed.ac.uk/s9810217/software/ipc/ipc_src.tar.gz}}\xspace}
\newcommand{\puffballsrctarballurl}{
\ahrefurl{\url{http://homepages.inf.ed.ac.uk/s9810217/software/puffball/puffball_dist.tar.gz}}\xspace}
\newcommand{\hydrusscripturl}{
\ahrefurl{\url{http://homepages.inf.ed.ac.uk/s9810217/software/ipc/condor_example/hydrus.sh}}\xspace}
\newcommand{\collatescripturl}{
\ahrefurl{\url{http://homepages.inf.ed.ac.uk/s9810217/software/ipc/condor_example/collate.ml}}\xspace}
\newcommand{\Condorexampletarballurl}{
\ahrefurl{\url{http://homepages.inf.ed.ac.uk/s9810217/software/ipc/condor_example.tar.gz}}\xspace}
\newcommand{\airbagpepamodelurl}{
\ahrefurl{\url{http://homepages.inf.ed.ac.uk/s9810217/software/ipc/condor_example/airbag.pepa}}\xspace}
\newcommand{\HaskellWebSite}{
\ahrefurl{\url{http://www.haskell.org}}}
\newcommand{\CygwinWebSite}{
\ahrefurl{\url{http://www.cygwin.org}}}
%%%% End of Url definitions %%%%%%%%%%%


%%%%%%%%%%%%%%%%%%%%%%%%%%%%%%%%%%%%%%%%%%%%%%%%%%%%%%%%%%%%%%%%%%%%%%%%
\newcommand{\QUESTION}[1]{\textbf{\ (>> \uppercase{#1} ??)\ }}
\newcommand{\CHECKTHIS}{\textbf{\ (<<< CHECK~THIS !!!)\ }}
\newcommand{\TODO}[1]{\textbf{\ (>> \uppercase{TODO: } #1 ??)\ }}
%%%%%%%%%%%%%%%%%%%%%%%%%%%%%%%%%%%%%%%%%%%%%%%%%%%%%%%%%%%%%%%%%%%%%%%%


% A command-line sequence
\newenvironment{commandline}{\begingroup\texttt{}{}\endgroup}

%


\title{A Tutorial in Using Ipc}
\author{
Allan Clark, Jeremy Bradley and Stephen Gilmore
}


\begin{document}
\maketitle

\newcommand{\ipccommandline}[1]{

\noindent
\texttt{#1}
}


\section{Introduction}
\pepa\cite{pepa} is a poplular stochastic process algebra.
The \ipc\ software tool is a compiler for \pepa\ models
into a format suitable for analysis by the \hydra\cite{pepahydra}
tool.
This tutorial will guide you through using the the \ipc\ software
tool to perform various measurements over various models.
For full documentation on the \ipc\ tool please refer to the manual.

\section{The Basics}
We begin with a basic model and will subsequently augment the
model to take into account more interesting features of \pepa\ and to
allow some more sophisticated measurements.
Our model is intended to represent a typical workplace coffee-room.
The first model is a straight-line model representing only a single tea-drinker.
The concrete syntax source for this first model is given in Figure
\ref{figure:pepafile:simpleTea}.
The rates are described in table \ref{table:pepa-rates:simpleTea}.

\begin{figure}[htb]
\verbatiminput{models/simpleTea.pepa}
\caption{
\label{figure:pepafile:simpleTea}
The source for the simple tea drinker model.
}
\end{figure}



\begin{table}[htb]
\begin{tabular}{|l|l|l|}
\hline\\
Rate Name & Value & Meaning\\
\hline
r\_thirst & 0.01 & Thirsts once every 100 minutes\\
\hline
r\_boil  & 1.0 & The kettle takes one minute to boil\\
\hline 
r\_pour  & 10.0 & It takes 6 seconds to pour\\
\hline
r\_milk  & 6.0 & It takes 10 seconds to add the milk\\
\hline
r\_stir  & 10.0 & It takes 6 seconds to stir the tea\\
\hline
r\_drink & 0.1  & It takes 10 minutes to drink the tea\\
\hline
\end{tabular}
\caption{
\label{table:pepa-rates:simpleTea}
The meaning of the rates in the simpleTea \pepa\ model.
}
\end{table}

The simplest invocation of \ipc\ on this model would be the following command:

\ipccommandline{
ipc simpleTea.pepa
}

This will cause \ipc\ to compute the entire state space of the model and then
solve the resulting CTMC for its steady-state solution. The resulting output
looks like the following that shown in Figure~\ref{figure:steady-output:simpleTea}.

\begin{figure}
\verbatiminput{simpleTea-steady-output}
\caption{\label{figure:steady-output:simpleTea}
The output from the simplest invocation of \ipc\ over the simpleTea model
}
\end{figure}

Beyond a simple steady-state analysis, \ipc\ is capable of performing the following
kinds of analyses over \pepa\ models.

\begin{tabular}{ll}
Steady-state     & \ipcflag{steady}       \\
Passage-time     & \ipcflag{passage}      \\
Transient        & \ipcflag{transient}    \\
Count measures   & \ipcflag{count}        \\
Passage-end      & \ipcflag{passage-end}  \\
\end{tabular}

Each of these will be described in more detail as we apply each to
the first simple model.
Special probe components are used to specify complex
performance measurements however to begin with the user need not know about
these and use the simpler interface of specifying activities of interest.
In general \ipc\ will work out which of the kinds of analysis you wish to
perform via the other command-line arguments that you give, but provided
are the options shown in the above list.
We now describe each of these kinds of measurements in detail.


\subsection{Passage-time Measurements}

A passage-time measurement is used to measure between two events.
The user specifies a set of start actions, the observation of the model
performing any one of these actions will start the measurement.
The user also specifies a set of stop actions and the measurement is
terminated when the model performs any actions within that set.
Without further ado a possible passage-time measurement over our
simple model would be to measure the passage between a $thirst$
activity and a $stir$ activity.
This can be done with the following command line:

\ipccommandline{
ipc simpleTea.pepa --source thirst --target stir 
}

This will produce to comma separated value files corresponding to the
probability density function and the cumulative distribution function
of the specified passage, in this case between the occurrence of a
$thirst$ activity and an occurrence of the $stir$ activity.
These files can then be processed by \gnuplot. Instead if you wish to
create a viewable graph directly with \ipc\ you can add a 
\ipcflag{graph-output} flag to your command line. The graphs in
Figure~\ref{figure:simpleTea-passage-results-graphs}
where produced with the command-line:

\ipccommandline{
ipc simpleTea.pepa --source thirst --target stir
    \textcolor{red}{--graph-output pdf}
}

Other arguments to the \ipcflag{graph-output} flag can be found from


\ipccommandline{ipc --help | grep ``graph-output''}

\begin{figure}
\begin{tabular}{ll}
\includegraphics[scale=0.4]{models/simpleTea-cdf.pdf}
&
\includegraphics[scale=0.4]{models/simpleTea-pdf.pdf}
\end{tabular}
\caption{\label{figure:simpleTea-passage-results-graphs}
Two graphs showing the PDF and CDF functions for our first
passage-time analysis over the simple tea drinker model
}
\end{figure}

This graph shows the cummulative distribution function, this means
it maps the probability that the given passage has completed at
the point in time corresponding to the x-axis. In this particular
case we can see that the probability rises quickly before flattening
out close to pronbability $1$ (on a long enough timeline all teas are made).

Note that the time units given are arbitrary and may mean whatever the user
has decided. The time units reflect whatever units the activity rates
in the model have used. As such these should all be the same. So in the model
we have said that the kettle takes approximately one minute to boil and
given that rate as $1.0$. Therefore the time units in this model are minutes.
From the graph we see that after around 1.3 time units, or 1.3 minutes, or
$78$ seconds there is a probability of 50\% of having completed the passage.
So after $78$ seconds there is an equal probability that the passage will
have completed as there is that it has not completed. 
After about $5$ minutes there is a high probability
that the passage has completed.

Note that the measurement is from the completion of a $thirst$ activity
and the completion of a $stir$ activity. Therefore in this measurement
the time taking to perform the $thirst$ activity is irrelevant but the
time take to perform the $stir$ activity does matter.




\subsection{Steady-state Measurements}
A steady-state measurement analyses the long-term probability that 
the model is in a given state or set of states.

This first measurement analyses the probability that the $TeaDrinker$
is currently in the process of making their tea. To do this we start
the measurement at the $thirst$ activity and stop at the $stir$ activity.
Shown here is both the command used to invoke \ipc\ and the response
from the subsequent run of \hydra.
Note here the short forms of the \texttt{--source} and 
\texttt{--target} options have been used, these are \texttt{-s}
and \texttt{-t} respectively.

\noindent
\texttt{
ipc -s thirst -t stir \textcolor{red}{--steady} --run-hydra simpleTea.pepa
}
\begin{verbatim}
State Measure 'steady_measure'

  mean                  1.2271774918e-02
\end{verbatim}
This tells us that there is a very low probability, just over 1 percent,
that the user is currently making tea, this is intuitively correct since
generally the user is either enjoying a tea or not yet thirsty.
We can therefore also measure the probability that the $TeaDrinker$ has
finished their last cup of tea but is not yet ready for another.
To do this we measure the probability of being in any state between the
occurrence of the $drink$ activity and the occurrence of a $thirst$ activity
(in this case there is only one such state).

\noindent
\texttt{
ipc \textcolor{red}{-s drink -t thirst} --steady --run-hydra simpleTea.pepa
}
\begin{verbatim}
State Measure 'steady_measure'
  mean                 8.9793475007e-01
\end{verbatim}
This tells us that there is almost an 81 percent chance that the user is
in this state.


\subsection{Transient Measurements}
Transient measures are similar to passage time measurements but may be used
to measure the short-term rather than steady-state probabilities.
Our simple first model behaves similarly in the short-term than in the long-term
this is because there are no interacting components so there is no need for
the single straight line component to 'settle' into a steady-state.
A transient measurement is invoked with the \ipcflag{--transient} flag, below
is shown a sample command-line.
As with a passage-time measurements the results are written to the file
named \texttt{simpleTea.PT\_RESULTS}.

\noindent
\texttt{
ipc -s thirst -t stir \textcolor{red}{--transient} --run-hydra simpleTea.pepa
}

\subsection{Count Measurements}
Count measures are used to calculate the mean rate at which a given set of
activities occurs within a model. The mean rate an activity is performed
depends upon the mean rate at which the activity is enabled and the
long-term probability that the model is in a state in which the activity
is enabled.
For our simple model we could calculate the mean rate at which the
$TeaDrinker$ has a cup. This may be calculated with the following
command-line:

\noindent
\texttt{
ipc \textcolor{red}{--count-measure drink} --run-hydra simpleTea.pepa
\newline
Count Measure 'my\_count\_measure'
  mean                0.008979347501
}

These results tell us that it is quite a low rate in comparison to
the rates involved in making the tea, again as one would expect.
At a mean rate of approximately $0.009$ this means the tea-drinker
enjoys a cup every $1/0.009$ minutes, or once every $111$ minutes.
This in intuitively correct since the tea drinker 'thirsts' 
at a rate of once every $100$ minutes and the time taken to make and
drink the tea are both small by comparison.


\section{Intermediate}

We will now consider some more complex measurements.
For this we update our model to include a boiler which continuously
boils water for the tea drinkers.

The model used for these measurements is given in Figure:
\ref{figure:pepafile:boiler}.

\begin{figure}[htb]
\verbatiminput{models/boiler.pepa}
\caption{
\label{figure:pepafile:boiler}
The source for the tea drinker and boiler.
}
\end{figure}

\subsection{Probe Specifications}
Rather than specifying start and stop actions one can specify
a probe. We begin with an example first.
Suppose we wish to find out the probability that the boiler has boiled
three times in a row without any water being taken from it.
If this probability is high then we can conclude that the boiler is
inefficient since it will unnecessarily boil water which is not then
subsequently used.

The probe specification to ask this question is the following:
$((boil, boil, boil)/pour) : start, pour : stop$
This probe specifies that once we have witnessed three $boil$ activities
without witnessing a $pour$ activity then the measurement should $start$.
Once the measurement is started the a $pour$ activity will stop it.
We could perform a passage-time measurement on this model and probe however
this would tell us only the expected time to a $pour$ activity after witnessing
a sequence of three $boil$ activities. This is not what we want.
Instead we use a steady-state measure to calculate the probablity that we are
in any of the measured states.
Therefore our command-line is:

\noindent
\texttt{
ipc --probe "((boil, boil, boil)/pour) : start, pour : stop" 
$\backslash$ \newline \hspace*{2cm} --steady --run-hydra boiler.pepa
}



%%%%%%%%%%%%%%%%%%%%%%%%%%%%%%%%%%%%%%%%%%%%%%%%%%%%%%%%%%%%%%%%%%%%
\bibliographystyle{unsrt}
\bibliography{../common/pepa}
\end{document}

